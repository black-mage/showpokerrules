\documentclass[a4paper,12pt]{article}
\usepackage{graphicx}
\usepackage{enumitem}
\renewcommand{\labelenumi}{\thesubsection.\arabic{enumi}}


\title{Show poker rules}
\author{Version 5.0}


\begin{document}
\maketitle
\pagebreak
\tableofcontents

\pagebreak

\section{Introduction}
“Show poker is the highly entertaining game where people are dealt 4 cards and instead of bidding with chips declare standard poker hands by showing cards in their hand. The winner at each round of betting gets to be the player who doesn’t have to pay to see the next card on the board. Theres a flop dealt before any declarations occur, and then declarations proceed in sequence until no other player wants to declare a better hand. Then every player gets the opportunity, in sequence from the highest hand, to either pay to see the next card or to fold their hand. There are 8 cards dealt onto the board in all including the flop, and after the final card is dealt, there is no betting, but the highest declared and shown hand takes the pot. Of course there’s plenty of potential to not show your hand until the final card is dealt and take the entire pot at that point. Which leads to all sorts of complex tactical play.” 

\pagebreak
\section{Rules}
\subsection{Blinds}
\begin{enumerate}

\item
Blinds are specified as increasing in timed levels, except during single/final table phases of the tournament, where blinds are specified as increasing after a certain number of hands.
\item
A blind is placed into the pot by each player before any cards are dealt


\end{enumerate}

\subsection{Initial Deal}
\begin{enumerate}
\item 4 cards are dealt to each player face down.

\item A 3 card flop is then dealt, before any showing or declarations occur.
\item Once the 3rd card in the flop is dealt, all hands dealt to players who are not sat in their seats are deemed dead and become the muck as they are drawn together by the dealer.
\item A player displaying the wrong number of cards at the end of the deal will mean the hand is declared as a misdeal and aborted.
\item A player displaying too many cards after a declaration or a buying of a card has occurred will have his hand declared as dead at which point, it will be drawn into the muck.
\item A player with too few cards discovered after a declaration or a buying of a card must play the hand out with his reduced number of cards.
\item An exposed card on the first deal of a hand is used as the first card on the flop. The card that would have been used as the first card on the flop is dealt face down to the player who would have had the exposed card dealt to him.
\item If more than 1 card is exposed during the deal of hole cards, then the hand is declared a misdeal and aborted.
\end{enumerate}


\subsection{Declarations}

\begin{enumerate}
\item Players cannot play the board until a 5th card emerges on it.
\item Any number of holecards can be used as part of a declared hand.
\item Any number of table cards can be used as part of a declared hand.
\item Declarations begin from the senior hand declaration from the previous street or, if there was no previous senior hand, the player to the left of the button.
\item Misdeclarations of hands are permitted.
\item A false declaration of a hand becomes a "check" if challenged subsequently. If not it remains as the falsely declared hand.
\item A player declaring a hand he has the cards to make legally has a true declaration that stands in the event of a challenge, even where the hand is less than 5 cards.
\item A hand declared that is not a hand which is stronger than the previously declared hand within the same round of declarations is treated as a falsely declared hand, and fails to a challenge.
\item Declarations of "fold", "check" and "buy" are binding.
\item The first complete sentence declaring a valid hand is binding.
\item Movements of chips over the line are binding and are interpreted as an action to repeatedly check and pay to see the next card.
\item A declaration that is ambigiously specified is always assumed to be the lowest hand that the declaration could possibly refer to.
\item A declaration of "check" by the person with the senior declaration in the previous round of declarations is to be interpreted as a redeclaration of the previous hand declared.
\item A declaration of "a x" or "x" where x is a card index means that the person is to be interpreted as declaring "x high" with any subsequent kickers he declares.
\item A declaration of "xs" or "xes" where x is a card index refers to "a pair of x" with any other pair of kickers the player declares.
\item A declaration of "with the board" refers only to valid kickers from the board for the hand declared.
\item A declaration of "check" when the previous declaration included a statement of "with the board" means the new hand is interpreted as being declared with any new kickers available since the previous declaration.
\item A declaration of a 6 card hand is always a misdeclaration and fails if challenged subsequently.
\item Declaring another person's hand is prohibited.
\item The penalty for declaring another person's hand is a minimum 2 hand sitout on the first occasion whilst still paying blinds.
\item The penalty for declaring another person's hand is a minimum 8 hand sitout whilst still paying blinds.
\item Speech play is prohibited in pots involving more than 2 players who have not folded.
\item The penalty for speech play in pots involving more than 2 players who have not folded is a minimum 1 hand sitout whilst paying blinds on the first occasion.
\item The penalty for speech play in pots involving more than 2 players who have not folded is a minimum 8 hand 4 hand sitout whilst paying blinds on subsequent occasions.
\item Declarations out of turn are subject to a single warning, then a one hand sitout whilst paying blinds, then subsequently one more sitout per offence.

\end{enumerate}
\subsection{Buying}
\begin{enumerate}
\item Once every person has had one opportunity to check or declare or fold after a leading declaration is made, a round of buying the subsequent card commences.
\item The player who has declared the senior hand pays nothing for the next card.
\item The action to buy or fold begins from the player to the left of the player who has declared the senior hand.
\item The first 3 purchasable cards are bought at one blind.
\item The final two cards are paid at the price of two blinds.
\item Once every player has either folded their hand, is allin or has bought the next card, the next card in the deck is burned and the subsequent card is placed face up on the board.
\item Shown cards in any given hand remain on show for the entire hand to players who have not yet folded.
\item Rabbit Hunting is permitted.
\item Buying out of turn is subject to a single warning, then a one hand sitout whilst paying blinds, then subsequently one more sitout per offence.
\end{enumerate}

\subsection{Showdown}
\begin{enumerate}
\item At showdown players all declare their best hand.
\item When no one has declared a superior hand for 10 seconds, a 30 second clock is called.
\item Players with hands inferior to the leading hand have their hands killed at the end of this 30 second period unless they are voluntarily folded before this point.

\end{enumerate}

\subsection{General}
\begin{enumerate}
\item The rules currently in force are those published online at pokerwhatson.com/ShowPoker.htm unless specifically declared otherwise within these rules.
\item Unless specifically declared otherwise within these rules, if these rules are different to the rules published at pokerwhatson.com/ShowPoker.htm then the published rules take precedence over these rules where practical.
\item The tournament director interprets and implements these rules. The tournament director's decision on how a situation is to progress is final. Tournament directors who deviate excessively from these rules should not be chosen by players.
\item Suggestions for improvements or refinements to rules should be in the first instance sent to the contact link at pokerwhatson.com/ShowPoker.htm


\end{enumerate}

\section{Glossary}
\subsection{Streets}
\begin{enumerate}
\item "turn cards" refers to the three cards on the row beneath the flop.
\item "river cards" refers to the two cards on the bottom row, beneath the turn cards.
\end{enumerate}
\subsection{Declarations}
\begin{enumerate}
\item "wheel" refers to a 5 high straight.
\item "x full of y" refers to a full house including a set of x and a pair of y.
\item "boat" refers to a full house.
\item "broadway" refers to an ace high straight.
\item "nut flush" refers to a flush which cannot be beaten by another flush of the same suit unless it is a straight flush.
\item "the nuts" refers to the best possible hand given the visible cards on the table at the time of the declaration. It is a prohibited declaration and fails to any challenge.
\end{enumerate}
\end{document}